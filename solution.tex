\documentclass[a4paper]{article}
\usepackage{ctex}
\usepackage{xeCJK}
\usepackage{amsmath}
\usepackage{amsfonts}
\usepackage{amssymb}
\usepackage{graphicx}
\usepackage{colortbl}
\usepackage{fancyvrb}
\usepackage{longtable}
\usepackage{xcolor}
\usepackage[hidelinks]{hyperref}
\usepackage[affil-it]{authblk}
\usepackage[top = 1.0in, bottom = 1.0in, left = 1.0in, right = 1.0in]{geometry}
\usepackage{amsthm}

\setCJKfamilyfont{kai}{KaiTi_GB2312}
\newcommand{\kai}{\CJKfamily{kai}}

\setCJKfamilyfont{song}{SimSun}
\newcommand{\song}{\CJKfamily{song}}

\newcommand\spc{\vspace{6pt}}
\newcommand{\floor}[1]{\lfloor {#1} \rfloor}
\newcommand{\ceil}[1]{\lceil {#1} \rceil}
\newcommand*\chem[1]{\ensuremath{\mathrm{#1}}}

\newtheorem{theorem}{Theorem}[section]
\newtheorem{lemma}[theorem]{Lemma}
\newtheorem{problem}{例题}

\date{\today}
%\date{\yestoday}
\title{solution}
\author{$\mathcal Pyh$}

\begin{document}

\maketitle

%\kai

\song

\begin{problem}
  k-Maximum Subsequence Sum\footnote{Codeforces 280D}

  给出一个长度为n的序列A, m次操作,操作有如下两种:

  1.给出i,val,把$A_{i}$变成$val$。

  2.给出l,r,k,询问把区间$[l,r]$划分成不超过k个不相交的区间,这些区间中数的和的最大值。

  数据范围:$n,m\leq 10^5, k\leq 20$
\end{problem}

这里介绍一种$O(k^2(n+m)\log n)$的算法(下面将会介绍更优的算法)。

因为是单点修改,我们考虑怎么将两个子区间的信息合并到当前区间。

$k=1$ 的情况就是区间最大子段和,是一个经典问题,不再赘述。

考虑 $k=2$ 的情况。

我们发现选的两个子区间要么都不跨左右子区间的边界,要么一个跨了另一个没跨。

这就相当于在一个子区间中是最大后缀(前缀)和,在另一个子区间中是一端贴着边界,总共包含两个子区间的最大和。

这个相当好维护。那么能不能推广呢?

首先我们在每个子区间维护一个数组 $maxx[1..k]$,$maxx[i]$记录该区间划分成$i$个子区间的最大和。

显然首先当前区间划分成的 $k$ 段区间可以完全在左子区间和右子区间,直接统计答案即可。

然后假设选的所有区间不跨两个子区间的边界,我们可以枚举左子区间放了多少个区间,直接统计答案。

跨边界的情况怎么办呢?我们注意到跨边界的那个区间可以拆分成两个部分,一个在左子区间,一个在右子区间。

于是我们对于一个区间,维护一个信息,表示当前区间划分了 $k$ 个子区间,其中最右边的子区间是贴着右边界的最大和。

同理,维护一个最左边的子区间是贴着左边界的最大和(这些都是数组,因为一个子区间可能是放$1..k$中的任意多个区间)。

为了维护上面的这两个“贴着边界最大和”的信息,我们需要再次考虑跨过中点的情况:这次轻车熟路了,我们只需要维护两端都贴着边界的最大和即可。

而维护“两端都贴着边界最大和”的信息,我们可以直接通过左右子区间维护的上述信息维护出来。

因为计算维护的信息数组的每一项需要枚举左右区间各自放了多少个区间,所以区间合并的复杂度是$O(k^2)$的,整体的复杂度为$O(k^2(n+m)\log n)$。

这是一个很巧妙的算法,但是因为 $k^2$ 比较大而无法通过本题。下文我将会讲述这个题目的正确解法,也是使用了线段树,不过是作为辅助数据结构来使用的。



\end{document}
