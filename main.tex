\documentclass[landscape]{article}
\usepackage{ctex}
\usepackage{amsmath}
\usepackage{amsfonts}
\usepackage{amssymb}
\usepackage{graphicx}
\usepackage{colortbl}
\usepackage{fancyvrb}
\usepackage{longtable}
\usepackage{xcolor}
\usepackage[hidelinks]{hyperref}
\usepackage[affil-it]{authblk}
\usepackage[top = 1.0in, bottom = 1.0in, left = 1.0in, right = 1.0in]{geometry}
\usepackage{amsthm}

\newcommand\spc{\vspace{6pt}}
\newcommand{\floor}[1]{\lfloor {#1} \rfloor}
\newcommand{\ceil}[1]{\lceil {#1} \rceil}
\newcommand*\chem[1]{\ensuremath{\mathrm{#1}}}

\newtheorem{theorem}{Theorem}[section]
\newtheorem{lemma}[theorem]{Lemma}

\date{Latest Update : \today}
%\date{\yestoday}
\title{Process table}
\author{$Pyh$}

\begin{document}

\maketitle

\begin{longtable}{ccccccccccc}
  \hline
  Date & Name & Source & First submitted & Status & Algorithm\\
  \hline
  2017.5.20 & 天天爱跑步 & NOIP2016 & AC & AC & 线段树合并\\
  \hline
  2017.5.20 & 旅行 & bzoj3531 & AC & AC & 动态线段树+树链剖分\\
  \hline
  2017.5.20 & 安全路径 & USACO Jan09 & TL & AC & 可并堆\\
  \hline
  2017.5.20 & 道馆之战 & ZJOI2011 & AC & AC & 树链剖分+线段树\\
  \hline
  2017.5.24 & 楼房重建 & THU2012day1 & WA & AC & 线段树\\
  \hline
  2017.5.25 & ticket & TEST20170525 & AC & AC & dp单调队列\\
  \hline
  2017.5.25 & fire & TEST20170525 & WA & AC & DP\\
  \hline
  2017.5.25 & substract & TEST20170525 & & AC & DP\\
  \hline
  2017.5.28 & K seq & hihocoder1046 & WA & AC & 主席树\\
  \hline
  2017.5.28 & GERALD07 & CChef MAR14 & AC & AC & 动态树+主席树\\
  \hline
  2017.5.29 & Peaks加强版 & ONTAK2010 & AC & AC & Kruskal+主席树\\
  \hline
  2017.5.29 & 可持久化并查集加强版 & BZOJ3674 & WA & AC & 主席树+并查集\\
  \hline
  2017.5.29 & 可持久化并查集 & BZOJ3673 & AC & AC & 主席树+并查集\\
  \hline
  2017.5.30 & 陌上花开 & BZOJ3262 & CE & AC & 树套树\\
  \hline
  2017.6.01 & Kapita加强版 & Ontak2013 & AC & AC & 多项式+数论(难)\\
  \hline
  2017.6.04 & 谈笑风生 & cogs laugh & WA & AC & 动态开点线段树\\
  \hline
  2017.6.04 & CGCDSSQ & cf475D & WA & AC & ST表+二分\\
  \hline
  2017.6.04 & Peaks & ONTAK2010 & AC & AC & Kruskal+主席树\\
  \hline
  2017.6.08 & cf814A & codeforces & AC & AC & 循环\\
  \hline
  2017.6.08 & cf814B & codeforces & AC & AC & 循环\\
  \hline
  2017.6.08 & cf814C & codeforces & AC & AC & 预处理+循环\\
  \hline
  2017.6.08 & cf814D & codeforces & AC & AC & 树形dp $\mathcal O(n^2)$\\
  \hline
  2017.6.09 & cf814D & codeforces & WA & AC & 树形dp+平衡树+扫描线 $\mathcal O(n\log n)$\\
  \hline
  2017.6.09 & hdu1512 & hdu & ML & WA & 线段树合并\\
  & & & & & \color{red}{(拍了一万组没有出错,不知道怎么回事)}\\
  \hline
  2017.6.09 & bzoj1056 & bzoj & AC & AC & Splay\\
  \hline
  2017.6.09 & bzoj1862 & bzoj & AC & AC & Splay\\
  \hline
  2017.6.10 & 金矿 & POI2001 & WA & AC & 线段树,trick\\
  & & cogs256 & & & (纵向区间可以通过差分的前缀和计算)\\
  \hline
  2017.6.10 & 星星 & 平衡树试题 & AC & AC & 线段树简单题\\
  \hline
  2017.6.10 & 大数小数 & 平衡树试题 & AC & AC & 排序算法\\
  \hline
  2017.6.10 & 最接近的数 & 平衡树试题 & AC & AC & STL的应用\\
  \hline
  2017.6.11 & APIO寻路 & APIO & WA & WA & 很多算法11KB\\
  \hline
  2017.6.12 & equation & lesson3 & AC & AC & 中途相遇\\
  \hline
  2017.6.12 & kmp & lesson3 & AC & AC & hash\\
  \hline
  2017.6.12 & maximum & lesson3 & AC & AC & 线段树\\
  \hline
  2017.6.13 & 捉迷藏 & ZJOI2007 & AC & AC & 线段树\\
  & & & & & 点对之间的关系,可以转化成线段树上的信息来维护\\
  \hline
  2017.6.13 & lyz & POI2009 & AC & AC & 线段树\\
  & & & & & 线段树判断完美匹配\\
  \hline
  2017.6.13 & k-Maximum & cf280D & AC & AC & 线段树\\
  & Subsequence Sum & & & & 线段树模拟费用流\\
  \hline
  2017.6.15 & segment & heoi2013 & WA & AC & 李超线段树\\
  \hline
  2017.6.15 & hdu5634 & bestcoder & WA & AC & 区间取phi线段树\\
  \hline
  2017.6.15 & uoj228 & uoj & AC & AC & 区间取根号线段树\\
  \hline
  2017.6.24 & codeforces 444C & codeforces & WA & AC & 普通线段树\\
  \hline
  2017.7.1 & balloc & Usaco & AC & AC & 线段树+贪心\\
  \hline
  2017.7.1 & 树据结构 & 51nod(160分) & WA & AC & 线段树+树链剖分\\
  \hline
  2017.7.2 & 双倍回文 & shoi2011 & AC & AC & manacher+set/并查集\\
  \hline
  2017.7.2 & cf444E & codeforces & AC & AC & 并查集+Hall定理(踩标程)\\
  \hline
  2017.7.3 & 整数分解为2的幂V2 & 51nod(1280分) & TL & AC & dp\\
  \hline
  2017.7.3 & 周期串查询 & 51nod(160分) & AC & AC & 线段树+hash\\
  \hline
  2017.7.3 & 公共祖先 & 51nod(80分) & AC & AC & 线段树合并\\
  \hline
  2017.7.4 & 变系数非波那契 & 51nod(320分) & RE & AC & 线段树维护矩阵\\
  \hline
  2017.7.4 & 捡石子 & 51nod(160分) & AC & AC & 线段树\\
  \hline
  2017.7.5 & 帕斯卡小三角 & 51nod1488 & WA & AC & 李超线段树\\
  \hline
  2017.7.5 & Function & codeforces & AC & AC & 李超线段树\\
  \hline
  2017.7.5 & 方程最小值 & 51nod & 理性 & 愉悦 & 权值线段树\\
  \hline
  2017.7.5 & 数点涂色 & sdoi & AC & AC & LCT+线段树\\
  \hline
  2017.7.5 & 共价大爷游长沙 & uoj & WA & AC & LCT\\
  \hline
  2017.7.6 & 维护直径 & 原创 & & & LCT\\
  \hline
  2017.7.6 & 重组病毒 & loj & AC & AC & LCT\\
  \hline
  2017.7.6 & bzoj1180 & bzoj & 理性 & 愉悦 & LCT\\
  \hline
  2017.7.6 & bzoj1453 & bzoj & 理性 & 愉悦 & LCT\\
  \hline
  2017.7.6 & bzoj1969 & bzoj & 理性 & 愉悦 & LCT\\
  \hline
  2017.7.6 & bzoj2594 & bzoj & 理性 & 愉悦 & LCT\\
  \hline
  2017.7.6 & bzoj2843 & bzoj & 理性 & 愉悦 & LCT\\
  \hline
  2017.7.6 & bzoj2888 & bzoj & 理性 & 愉悦 & LCT\\
  \hline
  2017.7.6 & bzoj3282 & bzoj & 理性 & 愉悦 & LCT\\
  \hline
  2017.7.6 & bzoj3589 & bzoj & 理性 & 愉悦 & LCT\\
  \hline
  2017.7.6 & bzoj3779 & bzoj & 理性 & 愉悦 & LCT\\
  \hline
  2017.7.7 & diyiti & 集训 & WA(30pt) & AC & DP\\
  \hline
  2017.7.7 & dierti & 集训 & WA(20pt) & \color{pink}{WA(20pt)} & 半平面交\\
  \hline
  2017.7.7 & disanti & 集训 & WA(10pt) & \color{pink}{WA(10pt)} & 凸包\\
  \hline
  2017.7.8 & y & 集训 & WA(30pt) & \color{pink}{WA(30pt)} & 线段树+dp?\\
  \hline
  2017.7.8 & m & 集训 & AC & AC & 虚树+dijstra算法\\
  \hline
  2017.7.8 & dagon & 集训 & WA(0pt) & \color{pink}{WA(10pt)} & dp+优化?\\
  \hline
  2017.7.9 & dalao & 集训 & ML(20pt) & \color{pink}{WA(40pt)} & 三维数点\\
  \hline
  2017.7.9 & meal & 集训 & WA(40pt) & \color{pink}{WA(50pt)} & 博弈\\
  \hline
  2017.7.9 & string & 集训 & WA(0pt) & \color{pink}{WA(0pt)} & 回文树\\
  \hline
  2017.7.9 & hdu4630 & hdu & lixing & yuyue & SegmentTree\\
  \hline
  2017.7.9 & cf600E & cf & AC & AC & dsu on tree\\
  \hline 
  2017.7.10 & cf570D & cf & AC & AC & dsu on tree\\
  \hline
  2017.7.10 & cf246E & cf & WA & AC & dsu on tree\\
  \hline
  2017.7.12 & 叶氏筛法 & loj & MLE & AC & 州阁筛\\
  \hline
  2017.7.13 & kth & 集训 & ML(20pt) & AC & 树状数组套线段树\\
  \hline
  2017.7.13 & prime & 集训 & AC & AC & 州阁筛\\
  \hline
  2017.7.15 & t1 & test & WA & AC & dp\\
  \hline
  2017.7.15 & t2 & test & AC & AC & 主席树\\
  \hline
  2017.7.15 & t3 & test & TL & AC & kmp\\
  \hline
  2017.7.16 & cf827F & cf & 理性 & 愉悦 & dp\\
  \hline
  2017.7.18 & bzoj3289 & bzoj & TL & AC & 分块+可持久化树状数组(在线)/莫队+树状数组(离线)\\
  \hline
  2017.7.18 & bzoj1086 & bzoj & AC & AC & 树分块\\
  \hline
  2017.7.18 & bzoj2038 & bzoj & RE & AC & 莫队算法\\
  \hline
\end{longtable}

\end{document}
