\documentclass[landscape]{article}
\usepackage{ctex}
\usepackage{amsmath}
\usepackage{amsfonts}
\usepackage{amssymb}
\usepackage{graphicx}
\usepackage{colortbl}
\usepackage{fancyvrb}
\usepackage{longtable}
\usepackage{xcolor}
\usepackage[hidelinks]{hyperref}
\usepackage[affil-it]{authblk}
\usepackage[top = 1.0in, bottom = 1.0in, left = 1.0in, right = 1.0in]{geometry}
\usepackage{amsthm}

\newcommand\spc{\vspace{6pt}}
\newcommand{\floor}[1]{\lfloor {#1} \rfloor}
\newcommand{\ceil}[1]{\lceil {#1} \rceil}
\newcommand*\chem[1]{\ensuremath{\mathrm{#1}}}

\newtheorem{theorem}{Theorem}[section]
\newtheorem{lemma}[theorem]{Lemma}

\date{Latest Update : \today}
%\date{\yestoday}
\title{Process table}
\author{$Pyh$}

\begin{document}

\maketitle

\begin{longtable}{ccccccccccc}
  \hline
  Date & Name & Source & First submitted & Status & Algorithm\\
  \hline
  2017.5.20 & 天天爱跑步 & NOIP2016 & AC & AC & 线段树合并\\
  \hline
  2017.5.20 & 旅行 & bzoj3531 & AC & AC & 动态线段树+树链剖分\\
  \hline
  2017.5.20 & 安全路径 & USACO Jan09 & TL & AC & 可并堆\\
  \hline
  2017.5.20 & 道馆之战 & ZJOI2011 & AC & AC & 树链剖分+线段树\\
  \hline
  2017.5.24 & 楼房重建 & THU2012day1 & WA & AC & 线段树\\
  \hline
  2017.5.25 & ticket & TEST20170525 & AC & AC & dp单调队列\\
  \hline
  2017.5.25 & fire & TEST20170525 & WA & AC & DP\\
  \hline
  2017.5.25 & substract & TEST20170525 & & AC & DP\\
  \hline
  2017.5.28 & K seq & hihocoder1046 & WA & AC & 主席树\\
  \hline
  2017.5.28 & GERALD07 & CChef MAR14 & AC & AC & 动态树+主席树\\
  \hline
  2017.5.29 & Peaks加强版 & ONTAK2010 & AC & AC & Kruskal+主席树\\
  \hline
  2017.5.29 & 可持久化并查集加强版 & BZOJ3674 & WA & AC & 主席树+并查集\\
  \hline
  2017.5.29 & 可持久化并查集 & BZOJ3673 & AC & AC & 主席树+并查集\\
  \hline
  2017.5.30 & 陌上花开 & BZOJ3262 & CE & AC & 树套树\\
  \hline
  2017.6.01 & Kapita加强版 & Ontak2013 & AC & AC & 多项式+数论(难)\\
  \hline
  2017.6.04 & 谈笑风生 & cogs laugh & WA & AC & 动态开点线段树\\
  \hline
  2017.6.04 & CGCDSSQ & cf475D & WA & AC & ST表+二分\\
  \hline
  2017.6.04 & Peaks & ONTAK2010 & AC & AC & Kruskal+主席树\\
  \hline
  2017.6.08 & cf814A & codeforces & AC & AC & 循环\\
  \hline
  2017.6.08 & cf814B & codeforces & AC & AC & 循环\\
  \hline
  2017.6.08 & cf814C & codeforces & AC & AC & 预处理+循环\\
  \hline
  2017.6.08 & cf814D & codeforces & AC & AC & 树形dp $\mathcal O(n^2)$\\
  \hline
  2017.6.09 & cf814D & codeforces & WA & AC & 树形dp+平衡树+扫描线 $\mathcal O(n\log n)$\\
  \hline
  2017.6.09 & hdu1512 & hdu & ML & WA & 线段树合并\\
  & & & & & \color{red}{(拍了一万组没有出错,不知道怎么回事)}\\
  \hline
  2017.6.09 & bzoj1056 & bzoj & AC & AC & Splay\\
  \hline
  2017.6.09 & bzoj1862 & bzoj & AC & AC & Splay\\
  \hline
  2017.6.10 & 金矿 & POI2001 & WA & AC & 线段树,trick\\
  & & cogs256 & & & (纵向区间可以通过差分的前缀和计算)\\
  \hline
  2017.6.10 & 星星 & 平衡树试题 & AC & AC & 线段树简单题\\
  \hline
  2017.6.10 & 大数小数 & 平衡树试题 & AC & AC & 排序算法\\
  \hline
  2017.6.10 & 最接近的数 & 平衡树试题 & AC & AC & STL的应用\\
  \hline
  2017.6.11 & APIO寻路 & APIO & WA & WA & 很多算法11KB\\
  \hline
  2017.6.12 & equation & lesson3 & AC & AC & 中途相遇\\
  \hline
  2017.6.12 & kmp & lesson3 & AC & AC & hash\\
  \hline
  2017.6.12 & maximum & lesson3 & AC & AC & 线段树\\
  \hline
  2017.6.13 & 捉迷藏 & ZJOI2007 & AC & AC & 线段树\\
  & & & & & 这种点对之间的关系,可以转化成线段树上的信息来维护\\
  \hline
\end{longtable}

\end{document}
