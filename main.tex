\documentclass[landscape]{article}
\usepackage{ctex}
\usepackage{amsmath}
\usepackage{amsfonts}
\usepackage{amssymb}
\usepackage{graphicx}
\usepackage{colortbl}
\usepackage{fancyvrb}
\usepackage{longtable}
\usepackage{xcolor}
\usepackage[hidelinks]{hyperref}
\usepackage[affil-it]{authblk}
\usepackage[top = 1.0in, bottom = 1.0in, left = 1.0in, right = 1.0in]{geometry}
\usepackage{amsthm}

\newcommand\spc{\vspace{6pt}}
\newcommand{\floor}[1]{\lfloor {#1} \rfloor}
\newcommand{\ceil}[1]{\lceil {#1} \rceil}
\newcommand*\chem[1]{\ensuremath{\mathrm{#1}}}

\newtheorem{theorem}{Theorem}[section]
\newtheorem{lemma}[theorem]{Lemma}

\usepackage{ccfonts}
\usepackage[T1]{fontenc}

\date{Latest Update : \today}
%\date{\yestoday}
\title{Process table}
\author{$\mathcal Pyh$}

\begin{document}

\maketitle

\begin{longtable}{ccccccccccc}
  \hline
  Date & Name & Source & First submitted & Status & Algorithm\\
  \hline
  2017.5.20 & 天天爱跑步 & NOIP2016 & AC & AC & 线段树合并\\
   & 旅行 & bzoj3531 & AC & AC & 动态线段树+树链剖分\\
   & 安全路径 & USACO Jan09 & TL & AC & 可并堆\\
   & 道馆之战 & ZJOI2011 & AC & AC & 树链剖分+线段树\\
  \hline
  2017.5.24 & 楼房重建 & THU2012day1 & WA & AC & 线段树\\
  \hline
  2017.5.25 & ticket & TEST20170525 & AC & AC & dp单调队列\\
   & fire & TEST20170525 & WA & AC & DP\\
   & substract & TEST20170525 & & AC & DP\\
  \hline
  2017.5.28 & K seq & hihocoder1046 & WA & AC & 主席树\\
   & GERALD07 & CChef MAR14 & AC & AC & 动态树+主席树\\
  \hline
  2017.5.29 & Peaks加强版 & ONTAK2010 & AC & AC & Kruskal+主席树\\
   & 可持久化并查集加强版 & BZOJ3674 & WA & AC & 主席树+并查集\\
   & 可持久化并查集 & BZOJ3673 & AC & AC & 主席树+并查集\\
  \hline
  2017.5.30 & 陌上花开 & BZOJ3262 & CE & AC & 树套树\\
  \hline
  2017.6.01 & Kapita加强版 & Ontak2013 & AC & AC & 多项式+数论(难)\\
  \hline
  2017.6.04 & 谈笑风生 & cogs laugh & WA & AC & 动态开点线段树\\
   & CGCDSSQ & cf475D & WA & AC & ST表+二分\\
   & Peaks & ONTAK2010 & AC & AC & Kruskal+主席树\\
  \hline
  2017.6.08 & cf814A & codeforces & AC & AC & 循环\\
   & cf814B & codeforces & AC & AC & 循环\\
   & cf814C & codeforces & AC & AC & 预处理+循环\\
   & cf814D & codeforces & AC & AC & 树形dp $\mathcal O(n^2)$\\
  \hline
  2017.6.09 & cf814D & codeforces & WA & AC & 树形dp+平衡树+扫描线 $\mathcal O(n\log n)$\\
  & hdu1512 & hdu & ML & WA & 线段树合并\\
  & & & & & \color{red}{(拍了一万组没有出错,不知道怎么回事)}\\
   & bzoj1056 & bzoj & AC & AC & Splay\\
   & bzoj1862 & bzoj & AC & AC & Splay\\
  \hline
  2017.6.10 & 金矿 & POI2001 & WA & AC & 线段树,trick\\
  & & cogs256 & & & (纵向区间可以通过差分的前缀和计算)\\
  & 星星 & 平衡树试题 & AC & AC & 线段树简单题\\
   & 大数小数 & 平衡树试题 & AC & AC & 排序算法\\
   & 最接近的数 & 平衡树试题 & AC & AC & STL的应用\\
  \hline
  2017.6.11 & APIO寻路 & APIO & WA & WA & 很多算法11KB\\
  \hline
  2017.6.12 & equation & lesson3 & AC & AC & 中途相遇\\
   & kmp & lesson3 & AC & AC & hash\\
   & maximum & lesson3 & AC & AC & 线段树\\
  \hline
  2017.6.13 & 捉迷藏 & ZJOI2007 & AC & AC & 线段树\\
  & & & & & 点对之间的关系,可以转化成线段树上的信息来维护\\
   & lyz & POI2009 & AC & AC & 线段树\\
  & & & & & 线段树判断完美匹配\\
   & k-Maximum & cf280D & AC & AC & 线段树\\
  & Subsequence Sum & & & & 线段树模拟费用流\\
  \hline
  2017.6.15 & segment & heoi2013 & WA & AC & 李超线段树\\
   & hdu5634 & bestcoder & WA & AC & 区间取phi线段树\\
   & uoj228 & uoj & AC & AC & 区间取根号线段树\\
  \hline
  2017.6.24 & codeforces 444C & codeforces & WA & AC & 普通线段树\\
  \hline
  2017.7.1 & balloc & Usaco & AC & AC & 线段树+贪心\\
   & 树据结构 & 51nod(160分) & WA & AC & 线段树+树链剖分\\
  \hline
  2017.7.2 & 双倍回文 & shoi2011 & AC & AC & manacher+set/并查集\\
   & cf444E & codeforces & AC & AC & 并查集+Hall定理(踩标程)\\
  \hline
  2017.7.3 & 整数分解为2的幂V2 & 51nod(1280分) & TL & AC & dp\\
   & 周期串查询 & 51nod(160分) & AC & AC & 线段树+hash\\
   & 公共祖先 & 51nod(80分) & AC & AC & 线段树合并\\
  \hline
  2017.7.4 & 变系数非波那契 & 51nod(320分) & RE & AC & 线段树维护矩阵\\
   & 捡石子 & 51nod(160分) & AC & AC & 线段树\\
  \hline
  2017.7.5 & 帕斯卡小三角 & 51nod1488 & WA & AC & 李超线段树\\
   & Function & codeforces & AC & AC & 李超线段树\\
   & 方程最小值 & 51nod & 理性 & 愉悦 & 权值线段树\\
   & 数点涂色 & sdoi & AC & AC & LCT+线段树\\
   & 共价大爷游长沙 & uoj & WA & AC & LCT\\
  \hline
  2017.7.6 & 维护直径 & 原创 & & & LCT\\
   & 重组病毒 & loj & AC & AC & LCT\\
   & bzoj1180 & bzoj & 理性 & 愉悦 & LCT\\
   & bzoj1453 & bzoj & 理性 & 愉悦 & LCT\\
   & bzoj1969 & bzoj & 理性 & 愉悦 & LCT\\
   & bzoj2594 & bzoj & 理性 & 愉悦 & LCT\\
   & bzoj2843 & bzoj & 理性 & 愉悦 & LCT\\
   & bzoj2888 & bzoj & 理性 & 愉悦 & LCT\\
   & bzoj3282 & bzoj & 理性 & 愉悦 & LCT\\
   & bzoj3589 & bzoj & 理性 & 愉悦 & LCT\\
   & bzoj3779 & bzoj & 理性 & 愉悦 & LCT\\
  \hline
  2017.7.7 & diyiti & 集训 & WA(30pt) & AC & DP\\
   & dierti & 集训 & WA(20pt) & \color{pink}{WA(20pt)} & 半平面交\\
   & disanti & 集训 & WA(10pt) & \color{pink}{WA(10pt)} & 凸包\\
  \hline
  2017.7.8 & y & 集训 & WA(30pt) & \color{pink}{WA(30pt)} & 线段树+dp?\\
   & m & 集训 & AC & AC & 虚树+dijstra算法\\
   & dagon & 集训 & WA(0pt) & \color{pink}{WA(10pt)} & dp+优化?\\
  \hline
  2017.7.9 & dalao & 集训 & ML(20pt) & \color{pink}{WA(40pt)} & 三维数点\\
   & meal & 集训 & WA(40pt) & \color{pink}{WA(50pt)} & 博弈\\
   & string & 集训 & WA(0pt) & \color{pink}{WA(0pt)} & 回文树\\
   & hdu4630 & hdu & lixing & yuyue & SegmentTree\\
   & cf600E & cf & AC & AC & dsu on tree\\
  \hline 
  2017.7.10 & cf570D & cf & AC & AC & dsu on tree\\
   & cf246E & cf & WA & AC & dsu on tree\\
  \hline
  2017.7.12 & 叶氏筛法 & loj & MLE & AC & 州阁筛\\
  \hline
  2017.7.13 & kth & 集训 & ML(20pt) & AC & 树状数组套线段树\\
   & prime & 集训 & AC & AC & 州阁筛\\
  \hline
  2017.7.15 & t1 & test & WA & AC & dp\\
   & t2 & test & AC & AC & 主席树\\
   & t3 & test & TL & AC & kmp\\
  \hline
  2017.7.16 & cf827F & cf & 理性 & 愉悦 & dp\\
  \hline
  2017.7.18 & bzoj3289 & bzoj & TL & AC & 分块+可持久化树状数组(在线)/莫队+树状数组(离线)\\
   & bzoj1086 & bzoj & AC & AC & 树分块\\
   & bzoj2038 & bzoj & RE & AC & 莫队算法\\
   & bzoj2038 & bzoj & AC & AC & 分块+预处理前缀和\\
  \hline
  2017.7.21 & pe561 & projecteuler & AC & AC & 数论\\
  \hline
  2017.7.22 & pe429 & projecteuler & AC & AC & 数论、组合数学、dp\\
  \hline
  2017.7.23 & bzoj数颜色 & 数颜色 & WA & AC & 带修改莫队\\
   & 三维偏序 & loj & AC & AC & cdq分治套线段树\\
  \hline
  2017.7.24 & poj2753 & poj & AC & AC & 半平面交\\
  \hline
  2017.7.25 & uva11178 & uva & AC & AC & 计算几何\\
  \hline
  2017.7.26 & poj2187 & poj & AC & AC & 旋转卡壳\\
  \hline
  2017.7.27 & cogs896 & cogs & WA & AC & 凸包\\
   & bzoj1004 & bzoj & AC & AC & polya\\
   & bzoj1005 & bzoj & AC & AC & 高精度+prufer+dp\\
   & bzoj1041 & bzoj & WA & AC & 数学\\
  \hline
  2017.8.8 & uoj80 & uoj & TL & TL & KM算法(被卡dfs)\\
   & uoj79 & uoj & TL & AC & 带花树算法\\
  \hline
  2017.8.9 & 挑战npc & wc & WA & AC & 带花树算法(需从后往前扫)\\
   & fortress & YLJX & AC & AC & 带花树算法(神构图)\\
   & tc-srm557-550 & tc & AC & AC & 二分图匹配\\
   & bzoj3997 & bzoj & AC & AC & 求最小链覆盖dp\\
   & uoj184 & uoj & TL & AC & 平面图分治\\
   & bzoj1006 & bzoj & RE & AC & 弦图求完美消除序列\\
  \hline
  2017.8.10 & uva10731 & uva & PE & AC & 强连通分量\\
   & hdu1269 & hdu & AC & AC & 强连通分量\\
   & poj2186 & poj & CE & AC & popular cows\\
   & zoj3316 & zoj & TL & AC & 带花树\\
  \hline
  2017.8.11-12 & sone1 & bzoj3513 & RE & AC & toptree(超难打)\\
  \hline
  2017.8.13 & 稳定婚姻 & 资料 & AC & AC & 完美匹配的关键边(水题)\\
   & compare & 资料 & AC & AC & 强连通分量(水题)\\
   & 排序 & 资料 & AC & AC & 最小反链覆盖(水题)\\
   & sell & 资料 & WA & AC & 矩阵乘法\\
   & eg0 & 资料 & 理性 & 愉悦 & 二分图匹配\\
   & eg1 & 资料 & 理性 & 愉悦 & 二分图匹配\\
   & eg2 & 资料 & 理性 & 愉悦 & 最小割\\
   & eg3 & 资料 & 理性 & 愉悦 & 二分图匹配\\
   & eg4 & 资料 & 理性 & 愉悦 & 最长反链\\
   & ex1 & 资料 & 理性 & 愉悦 & 二分图匹配\\
   & ex2 & 资料 & 理性 & 愉悦 & dp\\
   & ex3 & 资料 & 理性 & 愉悦 & 补图求最大匹配\\
   & ex4 & 资料 & 理性 & 愉悦 & 贪心/最小路径覆盖\\
   & extra\_ex1 & 资料 & 理性 & 愉悦 & 二分图匹配/分析性质\\
   & extra\_ex2 & 资料 & 理性 & 愉悦 & dp套二分图匹配\\
   & extra\_ex3 & 资料\\

  & & (1) & 理性 & 愉悦 & 关键边,交错环+交错偶路$O(m\sqrt{n})$\\

  & & (2) & 理性 & 愉悦 & 交错环+交错偶路$O(m\sqrt{n})$\\

  & & (3) & 理性 & 愉悦 & 关键点,交错偶路$O(m\sqrt{n})$\\

  & & (4) & 理性 & 愉悦 & 不存在边$O(n+m)$\\

  & & (5) & 理性 & 愉悦 & 二分图的最小字典序完备匹配$O(nm)$\\

  & & (6) & 理性 & 愉悦 & 二分图最大匹配$O(m\sqrt{n})$\\

  & & (7) & 理性 & 愉悦 & 一般图最大匹配$O(n^3)$\\
   & 城市交通 & 资料 & 理性 & 愉悦 & $(i,time)$跑最短路\\
   & 最右乘车 & 资料 & 理性 & 愉悦 & 最短路\\
   & 最小生成树 & 资料 & 理性 & 愉悦 & Kruskal+贡献\\
   & 新的开始 & 资料 & 理性 & 愉悦 & 最短路优化dp\\
   & 公路建设 & 资料 & 理性 & 愉悦 & 动态最小生成树\\
  \hline
  2017.8.14 & poj1149 & poj & AC & AC & 最大流\\
   & poj1637 & poj & AC & AC & 流量平衡\\
   & poj2391 & poj & AC & AC & 最大流\\
   & poj2699 & poj & 理性 & 愉悦 & 枚举+把边看成点+网络流\\
   & bzoj2753 & bzoj & WA & AC & 奇怪的最小生成树\\
   & bzoj1797 & bzoj & TL & AC & \href{http://blog.csdn.net/xy20130630/article/details/77162229}{详见博客}\\
  \hline
  2017.8.15 & zoj2676 & zoj & SF & AC & 二分+最小割\\
   & bzoj2400 & bzoj & WA & AC & 最小割\\
   & bzoj1497 & bzoj & AC & AC & 最小割\\
   & gift & 省队集训d10 & WA & AC & 最小割\\
  \hline
  2017.8.16 & poj3469 & poj & AC & AC & 最小割\\
   & happiness & 2015集训 & AC & AC & 最小割\\
   & chessboard & pc2 & WA & AC & 线段树优化连边+二分图关键边\\
   & bzoj1458 & bzoj & AC & AC & 最大流\\
   & bzoj3438 & bzoj & AC & AC & 最小割\\
   & loj6045 & loj & AC & AC & 最大权闭合子图\\
   & bzoj2132 & bzoj & TL & AC & 最小割\\
  \hline
  2017.8.17 & loj6058 & loj & TL & AC & 最小割(难)\\
  & noname & by\ pl & AC & AC & 最小割(水题)\\
  & loj6033 & loj & WA & AC & 二分图关键点\\
  \hline
  2017.8.18 & loj2146 & loj & WA & AC & 最大权闭合图\\
  & loj2042 & loj & WA & AC & 最小割树\\
  & loj2003 & loj & AC & AC & 二分+费用流\\
  & loj505 & loj & AC & AC & 最大流\\
  & bzoj3130 & bzoj & AC & AC & 二分+最大流\\
  & bzoj3511 & bzoj & AC & AC & 最小割\\
  \hline
  2017.8.20 & cf808F & cf & WA & AC & 二分+最小割\\
  & 腾讯狼人杀 & 计蒜课 & AC & AC & 二分+最小割\\
  \hline
  2017.8.21 & hit2543 & hit & AC & AC & 费用流\\
  & hit2715 & hit & AC & AC & 费用流\\
  & hit2739 & hit & TL & {\color{pink}TL} & 费用流\\
  & poj3680 & poj & TL & AC & 费用流\\
  & spoj-boxes & poj & AC & AC & 费用流\\
  & bzoj2597 & bzoj & AC & AC & 费用流\\
  & bzoj3171 & bzoj & AC & AC & 费用流\\
  & bzoj2245 & bzoj & WA & AC & 费用流\\
  \hline
  2017.8.23 & bzoj1520 & bzoj & WA & AC & 费用流\\
  & bzoj1930 & bzoj & WA & AC & 费用流\\
  & bzoj1449 & bzoj & AC & AC & 费用流\\
  & bzoj2895 & bzoj & AC & AC & 费用流\\
  & vijos1891 & vijos & AC & AC & 费用流\\
  & bzoj1283 & bzoj & AC & AC & 费用流\\
  & bzoj1061 & bzoj & AC & AC & 单纯形\\
  & bzoj3106 & bzoj & AC & AC & 单纯形\\
  \hline
  2017.8.24 & bzoj3265 & bzoj & TL & AC & 单纯形(对偶)\\
  & bzoj1937 & bzoj & TL & AC & 费用流\\
  & poj2175 & poj & TL & AC & 费用流消圈定理\\
  & bzoj1061 & bzoj & AC & AC & 费用流\\
  & bzoj3112 & bzoj & AC & AC & 费用流\\
  & csKUOHAO & cs & WA & AC & 模拟\\
  \hline
  2017.8.25 & uoj114 & uoj & & AC & 无源汇有上下界可行流\\
  & uoj115 & uoj & & AC & 有源汇有上下界最大流\\
  & uoj116 & uoj & & AC & 有源汇有上下界最小流\\
  \hline
  2017.8.26 & bzoj3698 & bzoj & AC & AC & 有上下界最大流\\
  & bzoj3876 & bzoj & AC & AC & 最小费用有上下界有源汇可行流\\
  & bzoj2324 & bzoj & AC & AC & 最小费用有上下界有源汇可行流\\
  \hline
\end{longtable}

\end{document}
